\documentclass[ngerman]{beamer}

\usepackage[german]{babel}
\usepackage[T1]{fontenc}
\usepackage{graphicx}
\usepackage{ifthen}
\usepackage{listings}
\usepackage{verbatim}
\usepackage{fancyhdr}
\usepackage{tikz}
\usepackage{dsfont}
\usetikzlibrary{spy,matrix}
\usepackage[style=mla, backend=bibtex]{biblatex}
\bibliography{storm}

\usepackage{xunicode}
\usepackage{xltxtra}
\usepackage{setspace}
\defaultfontfeatures{Mapping=tex-text}
\setmonofont[Mapping={}, Scale=MatchLowercase]{DejaVu Sans Mono}
\setsansfont[Scale=MatchLowercase]{Linux Biolinum O}
\setmainfont[]{Linux Libertine O}

\newbox\mytempbox
\newdimen\mytempdimen
\newcommand\includegraphicscopyright[3][]{%
  \leavevmode\vbox{\vskip3pt\raggedright\setbox\mytempbox=\hbox{%
  \includegraphics[#1]{#2}}%
    \mytempdimen=\wd\mytempbox\box\mytempbox\par\vskip1pt%
    \fontsize{3}{3.5}\selectfont{\color{black!25}{\vbox{\hsize=\mytempdimen#3}}}\vskip3pt%
}}

\newcommand\prelim[1]{\small }

\newcommand\strColor[1]{\color{beamer@blendedblue}{#1}}

\newcommand\sect[1]{\begin{center}\huge\strColor{#1}\end{center}}

\setbeamertemplate{navigation symbols}{}

\pagestyle{fancy}
\rhead{\vspace{.3em}\includegraphics[width=10em]{assets/FULogo_RGB.eps}}
\cfoot{}
\renewcommand{\headrulewidth}{0pt}

\lstset{language=C}
\lstset{basicstyle=\footnotesize\ttfamily}
\lstset{breaklines=true}
\lstset{keywordstyle=\color{purple}}

\usepackage{geometry}
\usepackage{tikz}
\usetikzlibrary{calc,arrows}

\newdimen\XCoord
\newdimen\YCoord

\newcommand*{\ExtractCoordinate}[1]{\path (#1); \pgfgetlastxy{\XCoord}{\YCoord};}%

\title{Gossiping on Mobile Devices}
\author{Michael Krause, Robin Nehls, Jakob Pfender}
\institute{FU Berlin\\Institut für Informatik\\Softwareprojekt
Mobilkommunikation}
\date{17. Dezember 2013}

\begin{document}

\frame{\titlepage}

\frame{
  \huge{Overview}

}

\frame{
\frametitle{Overview}
\begin{itemize}
  \item \textbf{Finished:}
    \begin{itemize}
      \item Gossip specification
      \item Gossip implementation
      \item Leader election specification
    \end{itemize}
  \item \textbf{Currently working on:}
    \begin{itemize}
      \item Leader election implementation
    \end{itemize}
\end{itemize}
}

\frame{
  \huge{Gossip implementation}

}

\frame{
    \frametitle{Gossip implementation}
    \begin{itemize}
      \item Nodes assign themselves random IDs, announce their presence,
        build neighbour tables and start gossiping messages to random
        neighbours
      \item Neighbour tables are kept up to date with cleanup functions
        and reannouncements
    \end{itemize}
}

\begin{frame}[fragile]
  \frametitle{Code snippets}
  % XXX Do we want code snippets?
  % XXX If yes, what's relevant?
\end{frame}

\frame{
  \huge{Leader election}

}

\frame{
    \frametitle{Leader election -- problems}
    \begin{itemize}
      \item Nodes only know their immediate neighbourhoods
      \item Exchanging neighbour tables is too much overhead
      \item Thus: ``Classic`` leader election not possible
    \end{itemize}
}

\frame{
    \frametitle{Leader election -- approach}
    \begin{itemize}
      \item If a node doesn't know who the leader is, it elects itself
      \item Any gossip communication carries ID of currently known
        leader in header
      \item If a node receives a leader ID that's ``better`` than the
        one it currently knows, it adapts that leader as new leader
      \item If a node receives a leader ID that's ``worse`` than the one it
        currently knows, it immediately replies with its own leader
      \item Leader information is thus disseminated through the network
        via gossiping
      \item Nodes always behave as if their locally known leader is the
        true leader
    \end{itemize}
}

\begin{frame}
 \frametitle{Discussion}
 \huge{Questions?}
\end{frame}

\end{document}
