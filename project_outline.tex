\documentclass[11pt,
  a4paper,
  ngerman,
  BCOR=7mm
]{scrartcl}
\usepackage[utf8]{inputenc}
\usepackage[ngerman]{babel}
\usepackage{tocbasic}
\usepackage[headsepline]{scrpage2}
\usepackage[german]{fancyref}
\usepackage{xcolor}
\usepackage[hyphens]{url}
\usepackage{listings}
\usepackage[pdftex]{graphicx}
\usepackage{courier}
\usepackage{amsmath}
\usepackage{dsfont}
\usepackage[colorlinks=false, linktocpage]{hyperref}
\setlength\parindent{0pt} 

\usepackage{DejaVuSansMono}
\lstset{language=C}
\lstset{basicstyle=\ttfamily}
\lstset{breaklines=true}
\lstset{keywordstyle=\color{purple}}

\usepackage{geometry}
\usepackage{tikz}
\usetikzlibrary{calc,arrows}

\newdimen\XCoord
\newdimen\YCoord

\newcommand*{\ExtractCoordinate}[1]{\path (#1); \pgfgetlastxy{\XCoord}{\YCoord};}%

\author{Michael Krause (), Robin Nehls (), Jakob Pfender (4282720)}
\title{Softwareprojekt Mobilkommunikation:\\Gossiping on Mobile
Devices}

\begin{document}

\maketitle

\newpage

\section*{Overview and Motivation}
\label{sec:overview_motivation}
Our task is to implement gossiping for RIOT on an application level,
such that it can ideally be used as a platform for any number of
applications that can benefit from information dissemination via
gossiping. Therefore, our overarching goal is to implement the gossiping
architecture in as modular a way as possible so that arbitrary
``plug-in`` applications can use it to share their information among
participating nodes. This approach also enables us to hand over the
completed code to the RIOT project after the software project has ended
in order to provide them with a generic gossiping platform upon which
further applications can be built.

\section*{Three Phases}
\label{sec:three_phases}
Our first goal is to implement a generic gossiping mechanism which
applications can use. In order to test this mechanism, we decided to
develop a simple application that can make good use of gossiping --
leader election. We will implement a gossiping-based algorithm that
allows a network of wireless nodes to elect and re-elect a leader node
based on arbitrary, preferably modular metrics. This implementation will
serve as a proof-of-concept that our gossiping implementation is viable.
The next step for us would be to build a complete application. We
decided to implement a time synchronization algorithm that makes use of
the leader election we implemented in the previous phase in order to
synchronize the clocks of the participating nodes to the leader's
clocks. Thus, the overall structure of our project looks like this:

\begin{itemize}
  \item \textbf{Phase 1}: Gossiping
  \item \textbf{Phase 2}: Leader Election
  \item \textbf{Phase 3}: Time Synchronization
\end{itemize}

\section*{Timeline}
\label{sec:timeline}
% \resizebox{\textwidth}{!}{%
%   \begin{tikzpicture}[snake=zigzag, line before snake = 5mm, line after snake = 5mm]
%   %draw horizontal line   
%   \draw (0,0) -- (2,0);
%   \draw[snake] (2,0) -- (5,0);
%   \draw (5,0) -- (7,0);
%   \draw[snake] (7,0) -- (15,0);
%   \draw (15,0) -- (17,0);
%   \draw[snake] (17,0) -- (20,0);
%   \draw (20,0) -- (34,0);
%   \draw[snake] (34,0) -- (41,0);
%   \draw (41,0) -- (43,0);
%   \draw[snake] (43,0) -- (46,0);

%   %draw vertical lines
%   \foreach \x in {0,2,5,7,15,17,20,34,41,43,46}
%      \draw (\x cm,3pt) -- (\x cm,-3pt);

%   %draw nodes
%   \draw (0,0) node[below=3pt] {$ 0 $} node[above=3pt] {};
%   \draw (2,0) node[below=3pt] {$ 2 $} node[above=3pt] {Milestone 1};
%   \draw (5,0) node[below=3pt] {$ 5 $} node[above=3pt] {};
%   \draw (7,0) node[below=3pt] {$ 7 $} node[above=3pt] {Milestone 2};
%   \draw (15,0) node[below=3pt] {$ 15 $} node[above=3pt] {};
%   \draw (17,0) node[below=3pt] {$ 17 $} node[above=3pt] {Milestone 3};
%   \draw (20,0) node[below=3pt] {$ 20 $} node[above=3pt] {};
%   \draw (34,0) node[below=3pt] {$ 34 $} node[above=3pt] {Milestone 4};
%   \draw (41,0) node[below=3pt] {$ 41 $} node[above=3pt] {};
%   \draw (43,0) node[below=3pt] {$ 43 $} node[above=3pt] {Milestone 5};
%   \draw (46,0) node[below=3pt] {$ 46 $} node[above=3pt] {};

%   \end{tikzpicture}
% }

\pgfmathsetmacro{\mintime}{0}
\pgfmathsetmacro{\maxtime}{13}
\newcommand{\timeunit}{Weeks}
\pgfmathtruncatemacro{\timeintervals}{13}
\pgfmathsetmacro{\scaleitemseparation}{4}
\pgfmathsetmacro{\timenodewidth}{2cm}
\newcounter{itemnumber}
\setcounter{itemnumber}{0}
\newcommand{\lastnode}{n-0}
% ============================= 

\newcommand{\timeentry}[2]{% time, description
\stepcounter{itemnumber}
\node[below right,minimum width=\timenodewidth] (n-\theitemnumber) at (\lastnode.south west) {#2};
\node[right] at (n-\theitemnumber.east) {};

\edef\lastnode{n-\theitemnumber}

\expandafter\edef\csname nodetime\theitemnumber \endcsname{#1}
}

\newcommand{\drawtimeline}{%
    \draw[very thick,-latex] (0,0) -- ($(\lastnode.south west)-(\scaleitemseparation,0)+(0,-1)$);
    \ExtractCoordinate{n-\theitemnumber.south}
    \pgfmathsetmacro{\yposition}{\YCoord/28.452755}
    \foreach \x in {1,...,\theitemnumber}
    {   \pgfmathsetmacro{\timeposition}{\yposition/(\maxtime-\mintime)*\csname nodetime\x \endcsname}
        %\node[right] at (0,\timeposition) {\yposition};
        \draw (0,\timeposition) -- (0.5,\timeposition) -- ($(n-\x.west)-(0.5,0)$) -- (n-\x.west);
    }
    \foreach \x in {0,...,\timeintervals}
    {   \pgfmathsetmacro{\labelposition}{\yposition/(\maxtime-\mintime)*\x}
        \node[left] (label-\x) at (-0.2,\labelposition) {\x\ \timeunit};
        \draw (label-\x.east) -- ++ (0.2,0);
    }   
}

\begin{tikzpicture}
\node[inner sep=0] (n-0) at (\scaleitemseparation,0) {};
% \timeentry{1.2}{}
\timeentry{1.5}{Milestone 1: Protocol description for gossiping}
% \timeentry{1.7}{}
\timeentry{3.1}{Milestone 2: Implementation of gossiping}
% \timeentry{3.3}{}
\timeentry{3.6}{Milestone 3: Protocol description for leader election}
% \timeentry{5.6}{}
\timeentry{6.6}{Milestone 4: Implementation of leader election}
% \timeentry{7.1}{}
\timeentry{7.4}{Milestone 5: Protocol description for time
synchronization}
\timeentry{13}{Milestone 6: Implementation of time synchronization}
\drawtimeline
\end{tikzpicture}


\end{document}

\end{document}
