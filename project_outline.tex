\documentclass[11pt,
  a4paper,
  ngerman,
  BCOR=7mm
]{scrartcl}
\usepackage[utf8]{inputenc}
\usepackage[ngerman]{babel}
\usepackage{tocbasic}
\usepackage[headsepline]{scrpage2}
\usepackage[german]{fancyref}
\usepackage{xcolor}
\usepackage[hyphens]{url}
\usepackage{listings}
\usepackage[pdftex]{graphicx}
\usepackage{courier}
\usepackage{amsmath}
\usepackage{dsfont}
\usepackage[colorlinks=false, linktocpage]{hyperref}
\setlength\parindent{0pt} 

\usepackage{DejaVuSansMono}
\lstset{language=C}
\lstset{basicstyle=\ttfamily}
\lstset{breaklines=true}
\lstset{keywordstyle=\color{purple}}

\usepackage{geometry}
\usepackage{tikz}
\usetikzlibrary{calc,arrows}

\newdimen\XCoord
\newdimen\YCoord

\newcommand*{\ExtractCoordinate}[1]{\path (#1); \pgfgetlastxy{\XCoord}{\YCoord};}%

\author{Michael Krause (), Robin Nehls (), Jakob Pfender (4282720)}
\title{Softwareprojekt Mobilkommunikation:\\Gossiping on Mobile
Devices\\Project Outline}

\begin{document}

\maketitle

\newpage

\section*{Overview and Motivation}
\label{sec:overview_motivation}
Our task is to implement gossiping for RIOT on an application level,
such that it can ideally be used as a platform for any number of
applications that can benefit from information dissemination via
gossiping. Therefore, our overarching goal is to implement the gossiping
architecture in as modular a way as possible so that arbitrary
``plug-in`` applications can use it to share their information among
participating nodes. This approach also enables us to hand over the
completed code to the RIOT project after the software project has ended
in order to provide them with a generic gossiping platform upon which
further applications can be built.

\section*{Three Phases}
\label{sec:three_phases}
Our first goal is to implement a generic gossiping mechanism which
applications can use. In order to test this mechanism, we decided to
develop a simple application that can make good use of gossiping --
leader election. We will implement a gossiping-based algorithm that
allows a network of wireless nodes to elect and re-elect a leader node
based on arbitrary, preferably modular metrics. This implementation will
serve as a proof-of-concept that our gossiping implementation is viable.
The next step for us would be to build a complete application. We
decided to implement a time synchronization algorithm that makes use of
the leader election we implemented in the previous phase in order to
synchronize the clocks of the participating nodes to the leader's
clocks. Thus, the overall structure of our project looks like this:

\begin{itemize}
  \item \textbf{Phase 1}: Gossiping
  \item \textbf{Phase 2}: Leader Election
  \item \textbf{Phase 3}: Time Synchronization
\end{itemize}

We further divide these phases into concrete milestones with completion
times attached to them.

\newpage

\section*{Milestones}
\label{sec:milestones}

\subsection*{Protocol description for gossiping -- 2-5 days}
\label{sub:protocol_gossiping}
 We aim to have a complete description of how our gossiping protocol is
 supposed to work finished before we start implementing it. This
 includes specifications for message formats and a generalized
 description of how the gossiping nodes will communicate. If possible,
 we will also design test cases during this phase.

\subsection*{Implementation of gossiping -- 2-10 days}
\label{sub:implementation_gossiping}
Once we have a formal specification of our gossiping protocol, we will
start implementing it. It is possible that we might have to adjust our
specifications if unforeseen issues arise during implementation.

\subsection*{Protocol description for leader election -- 2-5 days}
\label{sub:protocol_leader_election}
Once our gossiping platform is usable, we will develop specifications
for a leader election scheme based on it. An important aspect of this is
that users should be free to choose the metric(s) used for leader
election, so this aspect should be specified in a modular fashion.

\subsection*{Implementation of leader election -- 2-3 weeks}
\label{sub:implementation_leader_election}
We will implement a leader election application according to our
specifications developed in. As before, changes to the specification in
case of unforeseen issues are possible. An important aspect during
implementation will be the modularity of the leader election scheme in
regard to election metrics. The leader election application will serve
as a proof of concept that our gossiping platform is viable.

\subsection*{Protocol description for time synchronization -- 2-5 days}
\label{sub:protocol_time_synch}
Once we have achieved leader election via gossiping, we will develop
specifications for a time synchronization protocol that uses both the
gossiping platform and our leader election application to achieve
synchronization.

\subsection*{Implementation of time synchronization -- remaining time}
\label{sub:implementation_time_synchronization}
We will use the remaining project time to implement the time
synchronization application, as this task is potentially quite complex.

\newpage

\section*{Timeline}
\label{sec:timeline}
\pgfmathsetmacro{\mintime}{0}
\pgfmathsetmacro{\maxtime}{13}
\newcommand{\timeunit}{Weeks}
\pgfmathtruncatemacro{\timeintervals}{13}
\pgfmathsetmacro{\scaleitemseparation}{4}
\pgfmathsetmacro{\timenodewidth}{2cm}
\newcounter{itemnumber}
\setcounter{itemnumber}{0}
\newcommand{\lastnode}{n-0}

\newcommand{\timeentry}[2]{% time, description
\stepcounter{itemnumber}
\node[below right,minimum width=\timenodewidth] (n-\theitemnumber) at (\lastnode.south west) {#2};
\node[right] at (n-\theitemnumber.east) {};

\edef\lastnode{n-\theitemnumber}

\expandafter\edef\csname nodetime\theitemnumber \endcsname{#1}
}

\newcommand{\drawtimeline}{%
    \draw[very thick,-latex] (0,0) -- ($(\lastnode.south west)-(\scaleitemseparation,0)+(0,-1)$);
    \ExtractCoordinate{n-\theitemnumber.south}
    \pgfmathsetmacro{\yposition}{\YCoord/28.452755}
    \foreach \x in {1,...,\theitemnumber}
    {   \pgfmathsetmacro{\timeposition}{\yposition/(\maxtime-\mintime)*\csname nodetime\x \endcsname}
        \draw (0,\timeposition) -- (0.5,\timeposition) -- ($(n-\x.west)-(0.5,0)$) -- (n-\x.west);
    }
    \foreach \x in {0,...,\timeintervals}
    {   \pgfmathsetmacro{\labelposition}{\yposition/(\maxtime-\mintime)*\x}
        \node[left] (label-\x) at (-0.2,\labelposition) {\x\ \timeunit};
        \draw (label-\x.east) -- ++ (0.2,0);
    }   
}

\begin{tikzpicture}
\node[inner sep=0] (n-0) at (\scaleitemseparation,0) {};
\timeentry{1.5}{Milestone 1: Protocol description for gossiping}
\timeentry{3.1}{Milestone 2: Implementation of gossiping}
\timeentry{3.6}{Milestone 3: Protocol description for leader election}
\timeentry{6.6}{Milestone 4: Implementation of leader election}
\timeentry{7.4}{Milestone 5: Protocol description for time
synchronization}
\timeentry{13}{Milestone 6: Implementation of time synchronization}
\drawtimeline
\end{tikzpicture}

\section*{Further work}
\label{sub:further_work}
If we manage to finish the time synchronization application within the
alotted time and still have a reasonable amount of time left, we might
turn our focus to security aspects of our implementation -- i.e., how we
can achieve trust between gossiping nodes and how to deal with
attackers.


\end{document}

\end{document}
