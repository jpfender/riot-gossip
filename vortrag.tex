\documentclass[ngerman]{beamer}

\usepackage[german]{babel}
\usepackage[T1]{fontenc}
\usepackage{graphicx}
\usepackage{ifthen}
\usepackage{listings}
\usepackage{verbatim}
\usepackage{fancyhdr}
\usepackage{tikz}
\usepackage{dsfont}
\usetikzlibrary{spy,matrix}
\usepackage[style=mla, backend=bibtex]{biblatex}
\bibliography{storm}

\usepackage{xunicode}
\usepackage{xltxtra}
\usepackage{setspace}
\defaultfontfeatures{Mapping=tex-text}
\setmonofont[Mapping={}, Scale=MatchLowercase]{DejaVu Sans Mono}
\setsansfont[Scale=MatchLowercase]{Linux Biolinum O}
\setmainfont[]{Linux Libertine O}

\newbox\mytempbox
\newdimen\mytempdimen
\newcommand\includegraphicscopyright[3][]{%
  \leavevmode\vbox{\vskip3pt\raggedright\setbox\mytempbox=\hbox{%
  \includegraphics[#1]{#2}}%
    \mytempdimen=\wd\mytempbox\box\mytempbox\par\vskip1pt%
    \fontsize{3}{3.5}\selectfont{\color{black!25}{\vbox{\hsize=\mytempdimen#3}}}\vskip3pt%
}}

\newcommand\prelim[1]{\small }

\newcommand\strColor[1]{\color{beamer@blendedblue}{#1}}

\newcommand\sect[1]{\begin{center}\huge\strColor{#1}\end{center}}

\setbeamertemplate{navigation symbols}{}

\pagestyle{fancy}
\rhead{\vspace{.3em}\includegraphics[width=10em]{assets/FULogo_RGB.eps}}
\cfoot{}
\renewcommand{\headrulewidth}{0pt}

\lstset{language=C}
\lstset{basicstyle=\footnotesize\ttfamily}
\lstset{breaklines=true}
\lstset{keywordstyle=\color{purple}}

\usepackage{geometry}
\usepackage{tikz}
\usetikzlibrary{calc,arrows}

\newdimen\XCoord
\newdimen\YCoord

\newcommand*{\ExtractCoordinate}[1]{\path (#1); \pgfgetlastxy{\XCoord}{\YCoord};}%

\title{Gossiping on Mobile Devices}
\author{Michael Krause, Robin Nehls, Jakob Pfender}
\institute{FU Berlin\\Institut für Informatik\\Softwareprojekt
Mobilkommunikation}
\date{19. November 2013}

\begin{document}

\frame{\titlepage}

\frame{
  \huge{Project topic}

}

\frame{
\frametitle{Project topic}
\begin{itemize}
  \item Implement gossiping for RIOT-OS \scriptsize(Revolutionary Internet
    of Things OS)\normalsize\\on application level
  \item Used for keeping, disseminating and acting on information in a
    (P2P) network
  \item Nodes periodically pair off with random other nodes and share
    information
\end{itemize}
}

\frame{
  \huge{Objectives}

}

\frame{
    \frametitle{Objectives}
    \begin{itemize}
      \item Modularity
      \item Applications should be able to ``plug into`` gossiping layer
      \item Implement sample applications using our gossiping layer
    \end{itemize}
}

\frame{
  \huge{Work schedule}

}

\frame{
    \frametitle{Work schedule -- Three phases}
    \begin{itemize}
      \item \textbf{Phase 1:} Gossiping
      \item \textbf{Phase 2:} Leader Election
      \item \textbf{Phase 3:} Time Synchronization
    \end{itemize}
     
}

\frame{
  \huge{Milestones}

}

\frame{
    \frametitle{Milestones -- Gossiping}
    \textbf{Protocol description -- 2-5 days}
    \begin{itemize}
      \item Specification of message formats
      \item How do we communicate?
      \item Test cases
    \end{itemize}
    \textbf{Implementation -- 2-10 days}
    \begin{itemize}
      \item Implement according to specification
      \item Only adjust specification if we run into serious problems
    \end{itemize}
}

\frame{
    \frametitle{Milestones -- Leader election}
    \textbf{Protocol description -- 2-5 days}
    \begin{itemize}
      \item Define metrics for leader election
      \item Users should be able to choose their own metrics
      \item Modularity
    \end{itemize}
    \textbf{Implementation -- 2-3 weeks}
    \begin{itemize}
      \item Implement according to specification
      \item Only adjust specification if we run into serious problems
      \item Proof of concept for gossiping platform
    \end{itemize}
}

\frame{
    \frametitle{Milestones -- Time synchronization}
    \textbf{Protocol description -- 2-5 days}
    \begin{itemize}
      \item Use both gossiping platform and leader election
    \end{itemize}
    \textbf{Implementation -- remaining time}
    \begin{itemize}
      \item Probably very complex task
      \item Use remainder of project time to implement as much as
        possible
    \end{itemize}
}

\frame{
  \huge{Timeline}

}

\begin{frame}[fragile]
  \frametitle{Timeline}
    \pgfmathsetmacro{\mintime}{0}
    \pgfmathsetmacro{\maxtime}{13}
    \newcommand{\timeunit}{Weeks}
    \pgfmathtruncatemacro{\timeintervals}{13}
    \pgfmathsetmacro{\scaleitemseparation}{4}
    \pgfmathsetmacro{\timenodewidth}{2cm}
    \newcounter{itemnumber}
    \setcounter{itemnumber}{0}
    \newcommand{\lastnode}{n-0}

    \newcommand{\timeentry}[2]{% time, description
    \stepcounter{itemnumber}
    \node[below right,minimum width=\timenodewidth] (n-\theitemnumber) at (\lastnode.south west) {#2};
    \node[right] at (n-\theitemnumber.east) {};

    \edef\lastnode{n-\theitemnumber}

    \expandafter\edef\csname nodetime\theitemnumber \endcsname{#1}
    }

    \newcommand{\drawtimeline}{%
        \draw[very thick,-latex] (0,0) -- ($(\lastnode.south west)-(\scaleitemseparation,0)+(0,-1)$);
        \ExtractCoordinate{n-\theitemnumber.south}
        \pgfmathsetmacro{\yposition}{\YCoord/28.452755}
        \foreach \x in {1,...,\theitemnumber}
        {   \pgfmathsetmacro{\timeposition}{\yposition/(\maxtime-\mintime)*\csname nodetime\x \endcsname}
            \draw (0,\timeposition) -- (0.5,\timeposition) -- ($(n-\x.west)-(0.5,0)$) -- (n-\x.west);
        }
        \foreach \x in {0,...,\timeintervals}
        {   \pgfmathsetmacro{\labelposition}{\yposition/(\maxtime-\mintime)*\x}
            \node[left] (label-\x) at (-0.2,\labelposition) {\x\ \timeunit};
            \draw (label-\x.east) -- ++ (0.2,0);
        }   
    }

    \scalebox{0.7}{
      \begin{tikzpicture}
      \node[inner sep=0] (n-0) at (\scaleitemseparation,0) {};
      \timeentry{1.5}{Milestone 1: Protocol description for gossiping}
      \timeentry{3.1}{Milestone 2: Implementation of gossiping}
      \timeentry{3.6}{Milestone 3: Protocol description for leader election}
      \timeentry{6.6}{Milestone 4: Implementation of leader election}
      \timeentry{7.4}{Milestone 5: Protocol description for time
      synchronization}
      \timeentry{13}{Milestone 6: Implementation of time synchronization}
      \drawtimeline
      \end{tikzpicture}
    }
    
\end{frame}

\frame{
  \huge{Current progress}

}

\begin{frame}[fragile]
  \frametitle{Interfaces -- Data structures}
  \begin{lstlisting}
    typedef struct gossip_node {
      uint16_t id;              // ID of the node
      uint32_t last_send;       // Timestamp of last sent msg
      uint32_t last_recv;       // Timestamp of last received msg
    } gossip_node_t

    typedef struct gossip_node_list {
      size_t length;            // Length of neighbor list
      gossip_node* nodes;       // Neighbor list
    } gossip_node_list_t

    typedef enum gossip_strategy {      // Which node do we gossip to?
      RANDOM,
      OLDEST_FIRST
    } gossip_strategy_t
  \end{lstlisting}
\end{frame}

\begin{frame}[fragile]
  \frametitle{Interfaces -- Node functions}
  \begin{lstlisting}[basicstyle=\tiny\ttfamily]
    // Register message handler
    int gossip_init(void);

    // Register callback for received application messages
    int gossip_register_msg_handler(void (*handle) (void*, size_t));

    // Send HELLO message to broadcast address (on a regular basis) 
    int gossip_announce(void);

    // Get routing table
    gossip_node_list* gossip_get_all_neighbors(void);

    // Choose specific node according to some strategy
    gossip_node* gossip_get_neighbor(gossip_strategy_t strategy);

    // Send a message to some node
    int gossip_send(gossip_node, gossip_message);

    // Handle incoming message
    int gossip_handle_msg(void* msg, size_t length);
  \end{lstlisting}
\end{frame}

\begin{frame}
 \frametitle{Discussion}
 \huge{Questions?}
\end{frame}

\end{document}
